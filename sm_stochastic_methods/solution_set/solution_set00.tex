\documentclass{article}
\usepackage{amsmath, amssymb}
\usepackage{amsthm}

\title{Problem Set 0 Solutions}
\author{}
\date{}

\begin{document}
\maketitle

\section*{Problem 1}
\subsection*{(a) Describe the probability space for rolling two dice.}
Let the sample space be
\[
\Omega = \{(i, j) : 1 \leq i, j \leq 6\}
\]
where \( i \) and \( j \) represent the numbers on the first and second die, respectively. The total number of outcomes is \( 6 \times 6 = 36 \).

Since each outcome is equally likely, the probability of each outcome is
\[
P((i,j)) = \frac{1}{36}, \quad \text{for all } (i,j) \in \Omega.
\]

\subsection*{(b) Compute the probability of obtaining an even number.}
The possible sums of two dice are \( 2, 3, 4, \dots, 12 \). The even sums are \( 2, 4, 6, 8, 10, 12 \). We count the number of outcomes for each even sum:
\[
\begin{aligned}
    \text{Sum} = 2 &: (1,1) \quad \text{(1 outcome)} \\
    \text{Sum} = 4 &: (1,3), (2,2), (3,1) \quad \text{(3 outcomes)} \\
    \text{Sum} = 6 &: (1,5), (2,4), (3,3), (4,2), (5,1) \quad \text{(5 outcomes)} \\
    \text{Sum} = 8 &: (2,6), (3,5), (4,4), (5,3), (6,2) \quad \text{(5 outcomes)} \\
    \text{Sum} = 10 &: (4,6), (5,5), (6,4) \quad \text{(3 outcomes)} \\
    \text{Sum} = 12 &: (6,6) \quad \text{(1 outcome)} \\
\end{aligned}
\]
The total number of outcomes with even sums is \( 1 + 3 + 5 + 5 + 3 + 1 = 18 \).

Thus, the probability of getting an even sum is
\[
P(\text{even sum}) = \frac{18}{36} = \frac{1}{2}.
\]

\section*{Problem 2}
\subsection*{(a) Describe the probability space for selecting two balls without replacement.}
There are 6 balls labeled 1 to 6. The sample space consists of all possible pairs of balls selected without replacement:
\[
\Omega = \{(i, j) : 1 \leq i < j \leq 6\}.
\]
The number of possible outcomes is \( \binom{6}{2} = 15 \). Since each outcome is equally likely, the probability of each outcome is
\[
P((i, j)) = \frac{1}{15}.
\]

\subsection*{(b) Compute the probability of obtaining two balls with consecutive numbers.}
The pairs that correspond to consecutive numbers are: \( (1,2), (2,3), (3,4), (4,5), (5,6) \). There are 5 such pairs, so the probability of selecting two consecutive numbers is
\[
P(\text{consecutive numbers}) = \frac{5}{15} = \frac{1}{3}.
\]

\section*{Problem 3}
Let \( \Omega = \mathbb{R} \) and define
\[
\mathcal{A} = \{A \subset \mathbb{R} : A \text{ is countable}\} \cup \{A \subset \mathbb{R} : A^c \text{ is countable}\}.
\]

\subsection*{(a) Prove that \( \mathcal{A} \) is a \( \sigma \)-field.}
We need to verify the three properties of a \( \sigma \)-field:
\begin{enumerate}
    \item \( \mathbb{R} \in \mathcal{A} \): The set \( \mathbb{R} \) is uncountable, but \( \mathbb{R}^c = \emptyset \), which is countable. Thus, \( \mathbb{R} \in \mathcal{A} \).
    \item Closed under complements: If \( A \in \mathcal{A} \), either \( A \) is countable or \( A^c \) is countable. In both cases, \( A^c \in \mathcal{A} \).
    \item Closed under countable unions: Let \( A_1, A_2, \dots \in \mathcal{A} \). There are two cases:
    \begin{itemize}
        \item If each \( A_i \) is countable, then \( \bigcup A_i \) is countable, so \( \bigcup A_i \in \mathcal{A} \).
        \item If \( A_i^c \) is countable for each \( i \), then \( \bigcap A_i^c \) is countable, so \( \bigcup A_i \in \mathcal{A} \).
    \end{itemize}
\end{enumerate}
Thus, \( \mathcal{A} \) is a \( \sigma \)-field.

\subsection*{(b) Prove that \( (-\infty, 0] \notin \mathcal{A} \).}
The set \( (-\infty, 0] \) is uncountable, and its complement \( (0, \infty) \) is also uncountable. Hence, \( (-\infty, 0] \notin \mathcal{A} \).

\section*{Problem 4}
Let \( \Omega = \mathbb{N} \) and define
\[
\mathcal{A} = \{A \subset \mathbb{N} : A \text{ or } A^c \text{ is finite}\}.
\]

\subsection*{Show that \( \mathcal{A} \) is a field but not a \( \sigma \)-field.}
\begin{itemize}
    \item \( \mathcal{A} \) is a field because it is closed under finite unions, intersections, and complements.
    \item \( \mathcal{A} \) is not a \( \sigma \)-field because the infinite union \( \bigcup_{n=1}^{\infty} \{n\} = \mathbb{N} \notin \mathcal{A} \), since \( \mathbb{N}^c = \emptyset \) is finite.
\end{itemize}

\section*{Problem 5}
\subsection*{(a) Prove that the intersection of \( \sigma \)-fields is a \( \sigma \)-field.}
Let \( \{\mathcal{F}_i\}_{i \in I} \) be a collection of \( \sigma \)-fields. The intersection \( \bigcap_{i \in I} \mathcal{F}_i \) is a \( \sigma \)-field because:
\begin{itemize}
    \item It contains \( \Omega \),
    \item It is closed under complements, and
    \item It is closed under countable unions.
\end{itemize}

\subsection*{(b) Define the minimal \( \sigma \)-field containing \( \{1\} \) and \( \{2, 4\} \).}
Let \( \Omega = \{1, 2, 3, 4, 5, 6\} \) and define the sets \( A = \{1\} \) and \( B = \{2, 4\} \). The minimal \( \sigma \)-field containing \( A \) and \( B \) must contain:
\[
\emptyset, A, B, A^c, B^c, A \cup B, A \cap B, \Omega.
\]
The minimal \( \sigma \)-field is the collection of all subsets of \( \Omega \) formed by unions and complements of these sets.
\end{document}
